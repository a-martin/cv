\documentclass[a4paper, 10pt]{article}

%FONTS
\usepackage{fontspec}
\setmainfont[Mapping=tex-text, Numbers={Lowercase, Proportional}]{Linux Libertine O}
\setmainfont[Mapping=tex-text, Numbers={Lowercase,
  Proportional}]{Avenir Next Condensed}
% \setmainfont[Mapping=tex-text, Numbers={Lowercase, Proportional}]{Corbel}
%\setmainfont[Mapping=tex-text]{Hoefler Text}
\usepackage{xltxtra}

%PACKAGES
\usepackage{xcolor}
\usepackage{sectsty}
\usepackage{longtable}
\usepackage{hyperref}
\usepackage{multicol}
\usepackage{polyglossia}
\setmainlanguage[variant=usmax]{english}
\usepackage{csquotes}
\usepackage{ragged2e}

%MISE EN PAGE
\usepackage[left=1.5cm,top=1.25cm,right=1.5cm,bottom=1.5cm]{geometry}
\parskip 5pt
\parindent 0pt
\allsectionsfont{\mdseries}
\usepackage{enumitem}
\newcommand{\engquote}[1]{“#1”}
\newenvironment{exps}[1]%
 {\begin{list}{}%
         {\setlength{\leftmargin}{#1}}%
         \item[]%
 }
 {\end{list}}
\pagestyle{empty}
\def\changemargin#1#2{\list{}{\rightmargin#2\leftmargin#1}\item[]}
\let\endchangemargin=\endlist 


%COLORS
\definecolor{fire}{RGB}{178,34,34}
\definecolor{bleu}{RGB}{0,0,205}

%%%%%%%%%%%%%%%%%%%%%%

\begin{document}
\begin{center}

%NAME
{\Huge\fontspec{Avenir Next} Alexander \textsc{Martin}}

%PHONETIC TITLE
\smallskip
{\fontsize{19}{19}
\fontspec[Color=gray]{Linux Biolinum O}
[ˌʔæləɡˈze͜͠əndɚ ˈmɑɻʈ\,̚ʔn̩]}
\medskip
\hrule
\bigskip
%\hrule

%PERSONAL INFO
% Born: 31/03/1991
% in Westerville, USA\\
\smallskip
% {\fontspec[Color=lightgray]{Arial Unicode MS} ❦}\\
{\fontspec[Color=lightgray]{Arial Unicode MS} ■}\\
\smallskip
Laboratoire de Sciences Cognitives et Psycholinguistique\\
Département d'Études Cognitives \\ 
École Normale Supérieure -- PSL Research University\\
29, rue d'Ulm\\
75005 Paris, \textsc{France}\\
+33 (0) 1 44 32 29 82\\
\href{mailto:alexander.martin@ens.fr}{\texttt{alexander.martin@ens.fr}}\\
% \url{www.lscp.net/persons/martin/}

\end{center}



%EDUCATION
\section*{Education}

\begin{tabular}{c|l}
{expected June 30, 2017} & PhD in Cognitive Science\\
&{“Biases in phonological processing \& learning”}	\\
&École Normale Supérieure -- PSL Research University\\
&\\%%
2014 & Master's of Cognitive Science, \emph{with highest honors} (rank 1st/52)\\
&{Focus:} \emph{Experimental Linguistics}\\
&{“An Investigation into the Functional Load of Phonological Features and Perceptual Correlates”}	\\
&EHESS \textperiodcentered\ ENS \textperiodcentered\ Université Paris Descartes\\
&\\%%
2012 & Bachelor's of Language Sciences, \emph{with highest honors}\\
&{Focus:} \emph{Theoretical \& Descriptive Linguistics}\\
&Université Paris Diderot\\
&\\%%
2012 & Bachelor's of Applied Modern Languages, \emph{with highest honors}\\
&{Focus:} \emph{English \& German, Civilization Studies}\\
&Université Paris Diderot
\end{tabular}



%%%%%%%%%%%%%%%%%%%%
\begin{RaggedRight}

\section*{Current Projects}

\subsection*{Learning biases for phonetically natural rules}

The aim of this project is to examine different parameters that come into play
during the learning of different kinds of phonological rules
(specifically phonetically natural and typologically abundant rules
compared to unnatural, unattested ones), and the different levels at which a
learning bias can be observed. We explore this “naturality”
effect in different tasks (based both in perception and
production), and different populations, and by modulating
variability in the input, and exploring differential effects of
memory consolidation after sleep. We also study the role of bias
in transmission over time with the help of computer simulations.


\subsection*{Peceptual asymmetries in phonological processing}

This project focuses on asymmetrical processing of phonological
features during speech perception, both in prelexical phonological
processing, and in word recognition.  Its goal is to tease apart
lexical bias from low-level acoustic bias.  It includes a
computational component focusing on measuring lexical \emph{functional
load} to explain lexical influence on perception.


\subsection*{Phonological emergence in language contact}

As phonological contrasts can be lost over time, so too can they
emerge.  This project focuses on a current emergence in Dutch (namely
{\fontspec{Linux Biolinum O}/ɡ/}) with the specific aim of exploring the different levels at which
phonological emergence can occur (perception, production, the
lexicon), and what social (foreign language knowledge/use, education,
region) and linguistic (phonological and phonetic specificities of the
language) factors may be at play.


\end{RaggedRight}
%%%%%%%%%%%%%%%%%%%%


%Publications
\section*{Publications}

\subsection*{Submitted manuscripts}

\begin{itemize}

\item \textbf{Martin, A.}, \& S. Peperkamp (in revision). Sensitivity to
  phonetic naturalness in phonological rules: evidence from learning
  and consolidation with sleep.

\item \textbf{Martin, A.}, A. Guevara-Rukoz, T. Schatz, \& S. Peperkamp
  (resubmitted). Phonetic naturalness and the shaping of sound
  patterns: the role of learning bias and its transmission across
  generations.

\end{itemize}

\subsection*{Journal Articles}

\begin{itemize}
\RaggedRight

\item \textbf{Martin, A.}, \& S. Peperkamp (2017). Assessing
  the distinctiveness of phonological features in word recognition:
  prelexical and lexical influences. \textit{Journal of Phonetics},
  \textit{62}, 1--11.

\item \textbf{Martin, A.}, \& S. Peperkamp
  (2015). Asymmetries in the exploitation of phonetic
  features for word recognition.  \textit{The Journal of the
    Acoustical Society of America}, \emph{137}(4), \textsc{el}303--\textsc{el}314.


\item Fort, M., \textbf{A. Martin}, \& S. Peperkamp (2015). {Consonants are more important than vowels in the
    bouba-kiki effect}. \textit{Language and Speech}, \emph{58}(2), 247--266.

                                                         
\end{itemize}

\subsection*{Conference Proceedings}

\begin{itemize}
\RaggedRight

\item Fort, M., A. Weiss, \textbf{A. Martin}, \& S. Peperkamp ({2013}\nobreak\hspace{.05em}). \href{http://avsp2013.loria.fr/proceedings/papers/paper_41.pdf}{Looking
    for the bouba-kiki effect in prelexical infants}.  In: S. Ouni,
  F. Berthommier \& A. Jesse (eds.) \textit{Proceedings of the 12th
    International Conference on Auditory-Visual Speech
    Processing}, INRIA, 71--76.

\end{itemize}


\section*{Presentations}

\subsection*{Oral Presentations}

\begin{itemize}
\RaggedRight

\item  White, J., R. Kager, T. Linzen, \textbf{A. Martin}, A. Nevins,
  S. Peperkamp, K. Polgárdi, P. Rebrus, N. Topintzi, M. Törkenczy, \&
  R. van de Vijver (2017). \enquote{Artificial grammar learning of
    harmony patterns: Universal biases and L1 transfer}. Workshop at
  Societas Linguistica Europaea - 50th Annual Meeting, September
  10th--13th, 2017, Zürich, Switzerland (to be presented by J. White).

\item White, J., R. Kager, T. Linzen, \textbf{A. Martin}, A. Nevins,
  S. Peperkamp, K. Polgárdi, N. Topintzi, \&  R. van de Vijver
  (2017). \enquote{Preference for locality is affected by the
    prefix/suffix asymmetry: Evidence from artificial language
    learning}.  The 25th Manchester Phonology Meeting, May 25th--27th,
  2017, Manchester, United Kingdom (presented by J. White).

\item \textbf{Martin, A.}, M. van Heugten, R. Kager, \& S. Peperkamp
  (2016). \enquote{Phonological emergence in Dutch: Relating
    perception and production in contact-induced change}. Satellite
  workshop of LabPhon 15 on Marginal Contrasts, July 17th, 2016,
  Ithaca, New York.

\item \textbf{Martin, A.} \& S. Peperkamp (2016).  \enquote{Coalescing sources of bias in perception:
    Lexical and prelexical influences on the processing of phonological
    features}. The 15th Conference of
  Laboratory Phonology, July 13th--17th, 2016, Ithaca, New York.

\item Guevara-Rukoz, A., \textbf{A. Martin}, \& S. Peperkamp
  (2015). \enquote{The role of phonetic naturalness in shaping sound
    patterns}. Workshop on modeling variability in speech, Stuttgart,
  Germany, October 1st--2nd, 2015 (co-presented with
  A. Guevara-Rukoz).

\item \textbf{Martin, A.} (2015). \enquote{Similarity in the lexicon: A new measure of functional load \& experimental evidence}.  23rd
  Conference of the Student Organization of Linguistics in Europe
  (ConSOLE XXIII), Paris, France, January 7th--9th, 2015.

\item Fort, M., A. Weiss, \textbf{A. Martin}, \& S. Peperkamp (2013). \enquote{Looking
  for the bouba-kiki effect in prelexical infants}. 12th International
  Conference on Auditory-Visual Speech Processing, Annecy, France,
  August 29th--September 1st, 2013 (presented by M. Fort).

\end{itemize}

\subsection*{Poster Presentations}

\begin{itemize}

\item \textbf{Martin, A.}, M. van Heugten, R. Kager, \& S. Peperkamp (2016).
  \enquote{Relating perception and proudction in contact-induced
    change}. Workshop on Linking social effects in language processing
  to social effects in language evolution, September 15th--16th, 2016,
  Nijmegen, The Netherlands.


\item Peperkamp, S. \& \textbf{A. Martin}
  (2016). \enquote{Sleep-dependent consolidation in the learning of
    natural vs. unnatural phonological rules}. The 15th Conference of
  Laboratory Phonology, July 13th--17th, 2016, Ithaca, New York.

\item Fort, M., A. Weiss, \textbf{A. Martin}, \& S. Peperkamp (2013). \enquote{Looking for
  the bouba-kiki effect in prelexical infants}. Workshop on Infant
  Language Development, Donostia, Spain, June 20th--22nd, 2013
  (presented by S. Peperkamp).

\item Fort, M., A. Weiss, \textbf{A. Martin}, \& S. Peperkamp
  (2013\nobreak\hspace{.05em}). \engquote{Looking for the
    bouba-kiki effect in prelexical infants}. International Child
  Phonology Conference 2013, June 10th--12th, 2013, Nijmegen, The
  {Netherlands} (presented by M. Fort).

\item Fort, M., \textbf{A. Martin}, \& S. Peperkamp (2013\nobreak\hspace{.05em}). \engquote{Consonants are more
    important than vowels for the maluma-takete effect}. The 11th
  International Symposium of Psycholinguistics, March 20th--23rd, 2013,
  Tenerife, {Spain} (co-presented with M. Fort).

\end{itemize}


\subsection*{Invited Talks}

\begin{itemize}

\item Université Paris Diderot, LingLunch (October, 2016)

\item Utrecht Univeristy, Experimental Linguistics Talks in Utrecht
  (ELiTU) (June, 2016)

\end{itemize}



\section*{Teaching (at the Department of Cognitive Science of the
  École Normale Supérieure)}



\begin{center}


\begin{tabular}{|p{16cm}|}
\hline

\begin{center}
{Introduction to Phonology (with M. Giavazzi)}\\
{Fall term 2014, 2015, 2016 \textperiodcentered{} 12-week graduate course}\\
\end{center}

\begin{itemize}[noitemsep]
\item{Helped select and design course content}

\item Focus on rule-based and OT frameworks (main readings:
  Hayes, 2008 and Kager, 2004)

\item Led section (2 hours per week)

\item Gave various lectures

\item Designed and graded all problem sets

\item Prepared and graded midterm evaluation
\end{itemize}\\

\hline
\end{tabular}

\vspace{0.5cm}

\begin{tabular}{|p{16cm}|}
\hline

\begin{center}
{Topics in Phonology (with A. Cristia)}\\
{Spring term 2015, 2016 \textperiodcentered{} 12-week graduate course}\\
\end{center}

\begin{itemize}[noitemsep]

\item Collection of modules on various topics in phonology (i.a.,
  tonal phonology, prosodic hierarchy, lexical organization)

\item TA duties (e.g., mini-quiz preparation and grading, attendance)

\item Taught module (two lectures) on \emph{Evolutionary
    Phonology} (Blevins, 2004) 

\item Taught module (two lectures) on phonolexical structure (Itô \&
  Mester, 2008; Albright, 2009)        

\end{itemize}\\

\hline
\end{tabular}



\end{center}


\section*{Service Activities}

\begin{itemize}
\RaggedRight

\item Co-creator of the \texttt{\#barbarplots} initiative to raise
  awareness about data visualization techniques, including the
  associated Kickstarter which raised over 3,400€ (2016)

\item Co-founder and treasuror of the DEC Life scientific and cultural
  association (2015--2016)

\item Official translator (English/French) of the Department of Cognitive Science at
  the ENS (from 2015, paid service)

\item Creation and organization of the Phonology Journal Club at the
  LSCP (with Yue Sun) (2014--2016)

\item Coordination of the lab-wide Writing Group at the LSCP (2014--2015)

\item Creation and organization of the department-wide Language Groups
  (2014--2016)

\item Ad-hoc reviewing for \emph{Linguistic Inquiry}, OCP


\end{itemize}

\section*{Grants \& Awards}

\begin{itemize}
\RaggedRight
\item Travel award to attend the 15th Conference of Laboratory
  Phonology, \$400 (2016)

\item PhD grant for \enquote{Life and Health Sciences} from 
  {PSL} Research University (2014--2017)

\item PhD grant offer from the \emph{Cerveau, Cognition, Comportement}
  Doctoral School (declined)




\end{itemize}


\section*{In the Media}

\begin{itemize}

\item Research featured in a popular science article in Scientific American: \enquote{Does an M Sound Round to You?} written by Anne Pycha (\url{http://www.scientificamerican.com/article/does-an-m-sound-round-to-you/
h})  (November, 2015)

\item Research featured in a popular science article from the
  newsmagazine Vice: \enquote{The Parts of Speech} written by Ben
  Richond (\url{motherboard.vice.com/read/the-parts-of-speech})
  (April, 2015)

\end{itemize}


%%%%%%%%%%%%%%%%%%%%


\iffalse

%PROFESSIONAL EXPERIENCE
\section*{Non-Academic Professional Experience}

\subsection*{Language Teaching}

\begin{tabular}{r|l}

09/2013 -- 03/2014 & 

{French Teacher} (adults) \textperiodcentered\ \textit{Ensemble en Français} \\

&\\

09/2010 -- 06/2011 &

{English Teacher} (children) \textperiodcentered\ \textit{Les Petits Bilingues} \\

\end{tabular}

\subsection*{Translation}

\begin{tabular}{r|l}

01/2015 -- \emph{present} & 

{Translator} (English--French) \textperiodcentered\
\textit{Département d'Études Cognitives, ENS} \\

&\\


07/2011 -- 03/2012 & 

{Translator} (English--French) \textperiodcentered\ \textit{Atypic Tourism} \\

&\\

12/2010 -- 06/2011 &

{Language Intern} (translation \& transcription) \textperiodcentered\ \textit{Agat Films \& Cie / Ex Nihilo} \\

&\\

08/2010 -- 09/2010 &

{Translation Intern} \textperiodcentered\ \textit{\textsc{Fortis} Linguistique} \\

\end{tabular}





%SKILLS
\section*{Skills}

%\begin{multicols}{2}

\subsection*{Computer Skills}

\begin{itemize}
\item \textit{Typesetting}: {\LaTeX{}, Adobe InDesign}
\item \textit{Programming and Data Analysis}: {Python, PHP (basic), R}
\item \textit{Acoustic Analysis and Manipulation}: {Praat, Adobe Audition, Audacity}
\item \textit{Video Editing}: {Adobe Premiere Pro}
\end{itemize}


\subsection*{Languages}

\begin{itemize}
\item Professional competence in \textbf{English}, \textbf{French},
  \textbf{German}
\item Casual competence in several other languages
\end{itemize}

\fi



\iffalse


\subsection*{Scientific Skills}

\begin{itemize}
\item Extensive experience testing adult participants
\item Design and implementation of online experiments and use of
  Amazon's Mechanical Turk
\item Audio and audiovisual stimuli recording and preparation
\item Some experience with infant testing

\end{itemize}


\subsection*{Languages}
\begin{tabular}{lcl}
\textbf{English} & -- & Mother tongue\\
\textbf{French} & -- & Native-like proficiency\\
\textbf{German} & -- & Advanced proficiency\\
\textbf{Swedish} & -- & Intermediate proficiency\\
\textbf{Portuguese} & -- & Conversational\\
\textbf{Hungarian} & -- & Elementary conversational\\
\textbf{French Sign Language} & -- & Elementary conversational\\
\textbf{Spanish} & -- & Basic\\
\textbf{Arabic} & -- & Theoretical knowledge
\end{tabular}

%Basic knowledge of various other languages: (\nobreak\hspace{.05em}LSF--French Sign Language, Arabic, Spanish, Hungarian, Portuguese)

%\end{multicols}


\section*{Personal Development}
{
\renewcommand{\arraystretch}{1.3}
\begin{tabular}{r|l}
2014 & First prize in a Hungarian poetry reading competition\\
from 2013 & Hungarian language courses at the Hungarian Institute of Paris\\
from 2013 & English-Portuguese language exchange\\
2013 & Development of an Arabic text-to-speech computer program\\
2012 & French Sign Language classes at the ENS\\
2012 & English-Arabic language exchange\\
2011 & Voice acting in English and French for a language textbook recording\\
2011 & Volunteer tutoring at Université Paris Diderot (French as a Foreign Langauge, Translation)\\
2010 & Intensive summer courses in Swedish Language at Folkuniversitetet in Stockholm\\
2010 & Volunteer translation for \textsc{Tradadev}, an association working with NGOs\\
%from 2008 & International travel (USA, Canada, Alaska, Europe, Tunisia, India, Japan, Australia)
\end{tabular}
}
\fi

\end{document}



